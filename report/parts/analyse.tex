
\section{Analyse}

L'étape d'analyse est primordiale dans tout projet de moyenne ou grande
envergure, ce projet de graphes rentre bien dans cette catégorie. En
effet, il nous est demandé de représenter des graphes (qui peuvent se
représenter de manière différente selon les besoins que l'on a) et de
leur appliquer quelques algorithmes (dont l'expression dépend également
de la représentation des graphes).

Nous devons également implémenter de quoi lire un graphe depuis une
entrée donnée, ainsi que d'écrire un graphe vers une sortie donnée.

Tout au long de la phase analytique, une phrase revenait dans notre
manière de faire : « Penser en amont pour mieux servir en aval »,
autrement dit, nous avons essayé d'organiser le code de façon à ce
qu'il soit maintenable plus tard et que nous puissions lui ajouter
des améliorations au fil du temps.

Pour donner un petit exemple, pour lire un graphe, nous utilisons une
méthode qui nous permet de le faire de plusieurs méthodes, et il est
aisé d'ajouter/de retirer une méthode à cette liste.

Suit le diagramme de classe de notre application, dont nous allons
expliciter quelques points particuliers (qui correspondent notemment
aux Design Patterns qui ont émergés lors de la conception et de
l'implémentation).
\newpage
\subsection{Diagramme de classes}

\begin{figure}[H]
  \includepdf[pages={{},-}]{misc/class_diagram.pdf}
  \caption{Diagramme de classe}
\end{figure}

\newpage

\subsection{Graphe -- API}

\subsubsection{API}

Pour concevoir l'encodage des graphes, nous avons pensé immédiatement
à nous abstraire de la représentation. Nous sommes donc arrivé à une
API\footnote{Application Programming Interface} relativement générale.

L'idée que l'utilisateur pourra manipuler les éléments constitutifs du
graphe (nœuds, arêtes), ainsi que récupérer des collections d'éléments
utiles (les voisins d'un nœud donné par exemple).

L'API est détaillée ci dessous :

\begin{figure}[H]
\begin{minted}{c++}
  City* addCity(string name);
  City* addCity(string name, double lat, double lon);
  
  City* getCity(unsigned int);
  
  unsigned int addRoad(double distance, City* start, City* end);
  unsigned int addTrain(double distance, City* start, City* end);
  unsigned int addPlane(double distance, City* start, City* end);
  
  std::vector<City*> neighbors(City*, GenericFilter<double>*);
  std::vector<City*> neighborsOfNeighbors(City*, GenericFilter<double>*);
  std::vector<City*> allNeighbors(City*, GenericFilter<double>*);
  
  pair<vector<City*>, double> shortestPath(
      City* start, City* end,
      GenericFilter<double>* filter,
      std::function<double (Edge<double>)> cost);
  
\end{minted}
\caption{Extrait de \texttt{src/Map.hh}}
\end{figure}

On notera que la plupart des méthodes de la classe Graph (voir le
fichier \texttt{src/lib/Graph.hh}) sont virtuelles et non
implémentées. Ceci nous force donc à hériter de \texttt{Graph} et à
écrire une vraie représentation de graphe (cf
\texttt{src/DirectedGraph.cc}). D'une certaine manière, nous avons
émulé le mécanisme d'interfaces présentes en Go ou en Java.

Un autre point à noter est que nous avons choisi une gestion très « à
  la main » de manipulation des objets. Il était très peu pratique, en
tant que développeur, de devoir garder des pointeurs ou des références
vers des objets, aussi nous avons choisi d'exposer plutôt
des \emph{ids}, et de \emph{retourner} des références ou des pointeurs
vers les objets du graphes lors que c'était utile.

\subsubsection{Templates}

Voici un extrait du header de la classe \texttt{Graph} :

\begin{figure}[H]
\begin{minted}{c++}
  template <typename V>
  class Graph : public Nameable {
\end{minted}
\caption{Extrait de \texttt{src/Graph.hh}}
\end{figure}

On remarquera le \texttt{template <typename V>}, qui nous permet
d'obtenir des graphes totalement polymorphes (au sens programmation
fonctionnelle du terme) en le type qui étiquette les
arêtes\footnote{Ou presque, les paramètre de types doivent implémenter
  une comparaison ainsi que les opérateurs de flux. Malheureusement,
  il n'y a pas de moyen propre d'encoder ça en C++ (à moins de faire
  des assertions/casts absolument immondes en comptant sur le
  compilateur pour lever des erreurs au compile time).}. Ainsi, nous
sommes complètement flexibles de ce côté là (on pourra imaginer plus
tard de la gestion de flots, ou de la gestion de graphes sociaux, qui
nous obligent à ajouter de l'information sur les arêtes).

\subsubsection{Graphe orienté -- Graphe non-orienté}

Au cours de l'analyse, nous nous sommes rendu compte que les graphes
que nous manipulons sont peu denses, c'est à dire que chaque sommet est
relié à très peu d'autres sommets (comparé au nombre total de sommets),
il nous est donc venu à l'esprit d'utiliser une représentation sous
forme de listes d'adjacence.

En effet, la représentation matricielle serait extrêmement coûteuse en
mémoire, de plus elle serait parsemée de « 0 » qui nous sont inutiles.

On peut noter qu'il serait aisé de représenter un graphe sous sa forme
matricielle, il suffirait de créer des classes implémentant l'interface
\texttt{Graph} indiquée ci-dessus.

Nous avons donc représenté un graphe sous forme de listes d'adjacence,
mais comme le graphe peut être un multigraphe, au lieu de relier les
sommets entre eux, nous avons choisi de relier chaque sommet à toutes
les arêtes dont il est une extrémité (nous pouvons donc récuperer
l'autre extrémité à travers cette arête). Comme expliqué plus haut,
pour se faciliter la tâche, nous représentons au maximum les objets par
leur \texttt{id}, c'est pourquoi la liste est indicée par les
identifiants des sommets, et chaque élément de la liste stocke un
couple \texttt{(Informations sur le sommet, Arêtes)}.

Un graphe orienté est donc implémenté de cette façon, pour les graphes
non-orientés par contre, il y a une \textit{contrainte} en plus :
lorsque nous faisons un lien du sommet \texttt{a} vers le sommet
\texttt{b}, il faut également créer un lien du sommet \texttt{b} vers
le sommet \texttt{a}. Ainsi, un graphe non-orienté peut être vu comme
un graphe orienté, auquel on aura rajouté une contrainte sur ses
arêtes. Analytiquement parlant, la représentation d'un graphe
non-orienté \textit{hérite} de celle d'un graphe orienté, nous avons
simplement à réimplémenter la méthode \texttt{addEdge} vue dans l'API.

\subsection{Map -- Façade}

Originellement, il était prévu de ne manipuler que des instances de
\texttt{Graph}. Il nous a fortement été conseillé d'utiliser une façade
pour permettre à l'utilisateur final d'intéragir directement avec des
cartes.

De tout ce que permet de faire l'API, une carte est simplement un graphe
non-orienté avec des arêtes de type \texttt{double}. C'est pour cela
que la classe \texttt{Map} se définit comme suit :

\begin{figure}[H]
\begin{minted}{c++}
  class Map : public UndirectedGraph<double>
\end{minted}
\caption{Extrait de \texttt{src/Map.hh}}
\end{figure}

Un autre outil pour la façade est l'utilisation d'une classe \texttt{City}
à la place de la classe \texttt{Vertex}, comme ça l'utilisateur manipule
une carte et les villes qui lui sont associées. Conceptuellement parlant,
une ville hérite d'un sommet.

\subsection{Unité de calcul (ComputationUnit) -- Singleton}

L'unité de calcul représente la boite à outil de notre programme, c'est
ici que sont implémentés les principaux algorithmes demandés.

L'unité a besoin d'un graphe, à partir duquel les calculs seront
effectués, c'est pourquoi il y a une agrégation entre ces deux classes.
De plus, il est assez aisé de remarquer qu'il n'y aura qu'une seule
unité de calcul (on pourra éventuellement changer le graphe sur lequel
elle travaille), c'est ici que le pattern \textbf{Singleton} entre en
jeu, pour nous assurer que nous n'avons qu'une seule instance de
l'unité de calcul.

Les algorithmes implémentés dans l'unité de calcul sont le parcours en
largeur, le parcours en profondeur ainsi que l'algorithme de Dijkstra
pour trouver le plus court chemin entre deux villes. Ils sont tous
paramétrés par une fonction de \textit{filtre}.

Un filtre prend en paramètre une arête et renvoit un booléen, pour
savoir s'il faut prendre l'arête en compte lors du déroulement de
l'algorithme. Cela sert par exemple si l'on souhaite connaitre toutes
les villes accessible par la route, depuis une ville donnée.
Il y a simplement à implémenter la fonction de filtre de façon à ce
qu'elle renvoit \texttt{true} si l'arête est une route :

\begin{figure}[H]
\begin{minted}{c++}
  bool filter_road(Edge<V>& e) {
      return (e.type() == EdgeType::Road);
  }
\end{minted}
\caption{Un exemple de filtre}
\end{figure}

Une particularité de l'algorithme permettant de trouver le plus court
chemin entre deux sommets est qu'il permet également de définir ce que
représente réellement le coût d'une arête entre deux sommets.

On peut par exemple décider que le coût représente la distance entre
les villes, on peut également vouloir que le coût soit défini en terme
de temps de parcours entre les sommets (temps = distance / vitesse, la
vitesse étant déterminée selon le type d'arête, que ce soit une route,
un train ou un avion).

Voici le code des exemples pour illuster ce propos :

\begin{figure}[H]
\begin{minted}{c++}
  // Cost function using distance between cities
  double distance_cost(Edge<double> e) { return e.distance(); }

  // Cost function using estimated time between cities (time = distance / speed)
  double time_cost(Edge<double> e) { return e.distance() / static_cast<double>(e.type()); }
\end{minted}
\caption{Un exemple de fonctions de coût}
\end{figure}

\subsection{Entrées et Sorties (E/S) -- Stratégie}

Parlons maintenant des entrées et sorties (E/S). Nous souhaitions
vraiment avoir le choix du type d'E/S que nous déploierons, aussi nous
avons choisi d'implémenter cette partie du projet comme un pattern
Stratégie.

Nous avons conçu les entrées et sorties de façon à pouvoir facilement
rajouter une méthode de lecture (ou une méthode d'écriture) d'un graphe,
il suffit simplement de créer une classe implémentant l'interface
\texttt{GraphInput} ou \texttt{GraphOutput} et en redéfinissant la
méthode composant l'interface (\texttt{input} ou \texttt{output}).

Pour le moment, nous avons à notre disposition une entrée par texte
brut, et une sortie au format Dot.


\subsection{Lecture}

Nous avons choisi de créer un petit format de graphe pour les lire en
entrée, un format intitulé \texttt{RawTextGraph}. Voici un exemple de
graphe représenté sous ce format :

\begin{figure}[H]
\begin{minted}{haskell}
  France          -- Nom du graphe
  4               -- Nombre de villes
  Paris           -- Villes
  Lyon            -- Chaque ville a un id entre 0 et (Nombre de villes - 1)
  Marseille       -- L'ordre importe
  Lille
  4               -- Nombre de liaisons entre les villes
  0 1 Road 500.0  -- IdVille1, IdVille2, Type, Distance
  1 2 Road 350.0
  0 1 Train 500.0
  0 3 Plane 200.0 -- Lien entre Paris et Lille en avion, 200km
\end{minted}
\caption{Exemple de graphe}
\end{figure}

Le format permet de représenter un graphe orienté de la même façon
qu'un graphe non-orienté, la différence se fera dans le code, lors du
chargement et l'utilisation du graphe.

Pour l'implémentation, nous avons rajouté le support des opérateurs de
flux, c'est à dire qu'au lieu, par exemple, d'utiliser
\texttt{in.input(fichier)}, nous pouvons faire directement
\texttt{in << fichier} (\texttt{in} étant une instance de classe
implémentant l'interface \texttt{GraphInput}), nous pouvons voir cela
comme « \texttt{in} reçoit le contenu de \texttt{fichier} ».


\subsection{Écriture}

Nous avons choisi d'utiliser le format de sortie \texttt{dot}, qui est
reconnu comme un standard pour la sortie de graphe, il a pour avantages
de supporter les graphes orientés ou non-orientés, valués ou non-valués,
et nous pouvons donner un libellé aux sommets et afficher différement
les arêtes, cela nous est utile ici pour représenter les différents
types de liaisons entre les villes (route, chemin de fer,
voie aérienne). De nombreux outils supportent ce format en entrée,
comme par exemple l'outil \texttt{Graphviz}.

Voici la sortie \texttt{dot} du graphe donné en exemple plus haut :

\begin{figure}[H]
\begin{minted}{haskell}
  digraph France {
    0 [label="Paris"];
    1 [label="Lyon"];
    2 [label="Marseille"];
    3 [label="Lille"];
    0 -> 1 [label="500"];
    0 -> 1 [color=blue, style=dotted, label="500"];
    0 -> 3 [color=red, label="200"];
    1 -> 2 [label="350"];
  }
\end{minted}
\caption{Exemple de graphe au format \texttt{dot}}
\end{figure}

\begin{figure}[H]
\begin{center}
\includegraphics[scale=0.5]{misc/sample_graph.png}
\end{center}
\caption{Visualisation du graphe à l'aide de l'outil \texttt{Graphviz}}
\end{figure}

Une route est représentée par une arête noire, un chemin de fer par une
arête bleue en pointillés et une voie aérienne par une arête rouge.


De même que pour la lecture, nous avons ajouté le support de l'opérateur
\texttt{>>} : \texttt{out >> fichier}, cette fois pour dire « nous
envoyons le contenu de \texttt{out} dans \texttt{fichier} »
