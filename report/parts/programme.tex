
\section{Programme}

\subsection{Jeux de tests}

Il y a plusieurs graphes de tests disponibles avec le projet, il sont
situés dans le dossier \texttt{tests/graphtests}, il sont tous au
format \texttt{RawTextGraph}.

Pour le moment, nous avons un seul test généré à la main, c'est le
fichier nommé \texttt{test.yolo} qui comporte 10 villes et 13 arêtes.
C'est celui utilisé comme test dans le premier livrable, comme
expliqué plus tard dans le rapport.

En plus de ce test, nous avons développé un petit outil permettant de
générer aléatoirement des graphes au format \texttt{RawTextGraph}.
L'outil, développé en Python, est situé dans le dossier \texttt{tests},
et se nomme \texttt{gengraphs.py}. Il prend en paramètres le nombre de
sommets et le nombre d'arêtes voulus, et génère un graphe avec des
arêtes placées aléatoirement, avec un type (Route, Train ou Avion)
également aléatoire.

Cet outil nous permet d'avoir un jeu de tests très évolué : beaucoup
de sommets et très peu d'arêtes, très peu de sommets et beaucoup
d'arêtes, etc... C'est utile pour tester la fiabilité de nos
algorithmes ainsi que leur rapidité, l'objectif étant de pouvoir
manipuler des graphes d'une taille assez grande, de façon à simuler
le fait que l'outil pourrait être utilisé « dans la vraie vie ».


\subsection{Compilation}

Le projet est séparé en deux parties distinctes :
\begin{itemize}
\item La librairie, qui contient l'API détaillée auparavant, dans le
dossier \texttt{src/lib}
\item Les jeux de tests, qui contiennent tous les tests disponibles
pour évaluer l'API, situés dans le dossier \texttt{src/examples}
\end{itemize}

Il est possible de compiler séparément librairie et jeu de tests,
sachant que les tests requièrent que l'API ait été compiléé pour se
compiler eux aussi.

Pour compiler simplement la librarie, il suffit de faire
\texttt{make lib}, et tous les fichiers objets se trouveront dans le
dossier \texttt{build}.

Pour compiler les tests, un simple \texttt{make} suffit, et chacun des
fichiers \texttt{.cc} contenus dans le dossier \texttt{src/examples}
sera compilé et son exécutable se retrouvera à la racine du projet.


\subsection{Exécution}

Pour exécuter les tests séparément, il suffit de lancer l'exécutable
généré lors de la compilation. Pour faciliter la tâche si plusieurs
tests ont été compilés, nous avons fait un petit script permettant de
lancer tous les tests à la suite.

Si l'outil \texttt{Graphviz} est installé, l'outil récupère tous les
fichiers au format \texttt{dot} et en fait des fichiers \texttt{svg}
qui peuvent ensuite être visualisés.
