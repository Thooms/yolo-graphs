
\section{Livrables}

\subsection{Étape 1}

Pour ce premier livrable, nous devons pouvoir créer un graphe en
mémoire, et effectuer un parcours en largeur de profondeur arbitraire
dessus -- en notant bien que « afficher le graphe entier » revient à
faire un parcours « jusqu'au bout ». Le fichier de test pour ce
livrable est \texttt{src/examples/main1.cc}.

Le code est relativement auto-explicatif, mais pour rappel, il
effectue dans l'ordre les choses suivantes :

\begin{itemize}
\item On définit un graphe étiqueté par des réels, ainsi qu'une classe
  d'entrée permettant de remplir le graphe à partir d'un fichier.
\item On remplit le graphe selon le contenu du fichier \texttt{test.yolo}.
\item On définit un singleton de calcul.
\item On affiche le résultat des différents parcours demandés.
\item On définit une classe de sortie permettant d'obtenir un fichier
  sous format Dot, pour finalement récupérer \texttt{truc.dot}.
\end{itemize}
